\section{Anforderungsanalyse}

\subsection{Lastenheft}

Es soll ein Prototyp für die Erkennung und Lokalisierung eines Kraftfahrzeuges mit (zu) hohen Lärmemissionen entwickelt werden. Dabei sollen innerorts bis \SI{50}{km/h} mindestens drei Fahrspuren, mit einer Option auf zwei weitere, abgedeckt werden. Der Detektor soll so konstruiert werden, dass er problemlos von einer einzelnen Person in einem handelsüblichen Rucksack verstaut und transportiert werden kann. Die mindestens verfügbare Einsatzdauer soll eine halbe Stunde betragen. Die im Einsatz erfassten Daten sollen innerhalb von zwei Sekunden an Peripheriegeräte gesendet werden, um dort von dem Benutzer kontrolliert werden zu können. Der Benutzer kann sich in einem Aktionsradius von maximal \SI{30}{m} befinden und soll die Möglichkeit der Fernwartung haben. Die Messtoleranz des Prototypen soll im Bereich von \SI{0,5}{m} und \SI{1}{m} liegen. Eine Fehlerwahrscheinlichkeit von 99\% soll erreicht werden. (\color{red}Den letzten Punkt bitte nochmal überdenken - Sinnprüfung\color{black}) Die Messung soll mit einer Messfrequenz von \SI{48}{kHz} bei 24 Bit durchgeführt werden.