\section{Pflichtenheft}

Bei der Erstellung des Pflichtenhefts wurde, basierend auf dem Lastenheft des Traffic-Noise--Detectors, auf die individuellen Forderungen des Auftraggebers eingegangen. Die daraus entstandenen Umsetzungen sind aus der folgenden Tabelle zu entnehmen:

\begin{center}
\begin{tabular}{|p{2cm}|p{6cm}|p{6cm}|}
\hline
\textbf{Ifd.-Nr} & \textbf{Anforderungen gemäß Lastenheft:} & \textbf{Umsetzung im Pflichtenheft:}\\
\hline
1 & Überwachung von drei Fahrspuren & Mit Hilfe von sechs Mikrophonen, welche durch den Auftraggeber bereitgestellt werden.\\
\hline
2 & Angemessenes Gewicht & Benötigte Halterungen/ Vorrichtungen werden bevorzugt mittels eines 3D-Druckverfahrens hergestellt, um so das Gewicht zu senken. \hfill\\
\hline
3 & Verstaubar in einem Rucksack & Ein entsprechendes Gehäuse wird durch den Auftraggeber zur Verfügung gestellt. \\
\hline
4 & Leicht aufbaubar & Als Unterbau wird ein Stativ aus dem Fotografie Bereich verwendet. Dieses verfügt bereits über eine passende Aufnahme für den Prototypen.\\
\hline
5 & Einsatzdauer > 0.5 Stunden & Die Energieversorgung wird mittels Akkumulatoren sichergestellt \\
\hline
6 & Übertragung der erfassten Daten in einem Aktionsradius von \SI{30}{m} & WLAN Netz \\
\hline
7 & Betreiben einer Fernwartung & WLAN Netz\\
\hline
8 & Messtoleranz im Bereich \SI{0,5}{m} und \SI{1}{m} & \\
\hline
9 & Fehlerwahrscheinlichkeit von 99\% & \\
\hline
10 & Die Messfrequenz soll \SI{48}{kHz} bei 24 Bit betragen \hfill & Das System wird entsprechend eingestellt \hfill \\
\hline
11 &&\\
\hline
12 &&\\
\hline
13 &&\\
\hline
14 &&\\
\hline
&&\\
\hline
&&\\
\end{tabular}
\end{center}