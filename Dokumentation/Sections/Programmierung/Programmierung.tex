\section{Programmierung}

\begin{tabularx}{\columnwidth}{|p{4cm}|X|}
	%	\hline
	%	\textbf{Abschnitt} & \textbf{Inhalt}\\
	\hline
	\textbf{Name} & Verbindung ROS und Controller\\
	\hline
	\textbf{Autor(en)} & Heiko Nöldeke \\
	\hline
	\textbf{Priorität} & hoch\\	
	\hline	
	\textbf{Kritikalität} & hoch\\
	%	\hline
	%	\textbf{Quelle} & \\
	\hline
	\textbf{Verantwortlicher} & Heiko Nöldeke\\
	\hline
	\textbf{Kurzbeschreibung} & \makecell[tl]{Mit diesem Test soll die fehlerfreie Kommunikation zwischen\\dem ROS und dem Controller getestet werden.} \\
	\hline
	\textbf{Auslösendes Ereignis} &  \makecell[tl]{Eingabe Konsole:\\rosrun joy joy\_node\\Eingabe in zweiter Konsole:\\rostopic echo joy}\\
	\hline
	\textbf{Akteure} & Raspberry, Xbox Controller, User\\
	\hline
	\textbf{Vorbedingung} & Konsole für Eingabe vorbereitet\\
	\hline
	\textbf{Nachbedingung} & Konsole für weitere Eingaben bereit\\
	\hline
	\textbf{Ergebnis} & \makecell[tl]{std\_msgs/Header header\\float32[] axes\\int32[] buttons}\\
	\hline
	%	\textbf{Hauptszenario} & \makecell[tl]{text}\\
	%	\hline
	%	\textbf{Alternativszenario} & \\
	%	\hline
	%	\textbf{Ausnahmeszenario} & \\
	%	\hline
	%	\textbf{Qualitäten} & \\
	%	\hline
\end{tabularx}
\captionof{table}{Test Case Verbindung ROS und Controller}
\label{tab:TestCaseROSController}

\newpage