\section{Umgebungssituation}

In verschiedenen Städten um den Globus gibt es das Problem, dass Automobile teilweise so manipuliert werden, dass sie die maximal zulässige Lautstärke überschreiten. Bisher müssen diese Lautstärkesünder manuell von Polizeistreifen ausfindig gemacht und kontrolliert werden. Ähnlich dem Konzept der Geschwindigkeitskontrolle, mit Messung und Beweisaufnahme durch ein Foto, soll mit dieser Entwicklung eines ersten Prototypen eine automatisierte Erkennung der Lautstärke erfolgen und für den Bußgeldbescheid aufbereitet werden.

Wichtige Anforderungen dafür sind auch, neben einer benutzerfreundlichen Bedienung eine Konstruktion, welche den Einsatz im Straßenverkehr ermöglicht. Dazu kommt vor allem eine gute Transportfähigkeit und Wasserbeständigkeit.

\newpage