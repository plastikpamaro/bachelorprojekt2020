\section{Zusammenfassung der Aufgabenstellung}

\begin{tabular}{|p{15cm}|}
\hline
\textbf{Project Charter:} Traffic-Noise-Detector (Lärmblitzer), Bachelor Project - Hr. Otto, Hr. Nöldeke, Hr. Bolsch, Hr. Roschkowski\\
\hline
\textbf{Mission:} Aufbau eines Prototypen zur Erkennung und Lokalisierung von Fahrzeigen mit (zu) hoher Lärm Emission (z. B. \url{https://www.auto-motor-und-sport.de/verkehr/laerm-blitzer-in-europa-fallen-gegen-auto-poser/}).\\
\hline
\textbf{Deliverables (incl. timing):}
\begin{itemize}
\item Anforderungsentwicklung (Mechanisch, Akustisch, Elektrisch, Algorithmisch)
\item Systematische Auswahl / ggf. Kombination oder Weiterentwicklung
\item Realisierung
\item Test und Bewertung der Eignung
\item Überarbeitung basierend auf den Testergebnissen [T0+6 Wochen]
\item Abschlussintegration, Demonstration/Vortag und Dokumentation [T0+12 Wochen]
\end{itemize}\\
\hline
\textbf{Expected Scope / Approach / Activities:}
\begin{itemize}
\item Einarbeitung in das Messsystem und die Programmierumgebung
\item Einarbeitung in den Stand der Technik von Algorithmen zur Erkennung und Lokalisierung akustischer Signale
\item Zielgerichtete Auswahl, Weiterentwicklung / Kombination im Hinblick auf genutzte Hardware sowie die Erkennung mit einer hohen Erkennungsrate
\end{itemize}\\
\hline
\textbf{Strategic alignment factors:}
\newline
Integration in dei Arbeitsgruppe Urban Mobility Lab mit den laufenden Arbeiten\\
\hline
\textbf{Timeframe/Duration:}
\begin{itemize}
\item Start 1.10.2020 (Vorbereitung
\item Abschluss 30.3.2021 (gerne früher)
\end{itemize}\\
\hline
\textbf{Team Resources:}
\newline
Nutzung Labor Stiftstraße 69 Raum 109 mit der dort verfügbaren Infrastruktur (Elektronikentwicklung, Software Entwicklung, Server, Löteinrichtung; Kamera, Testfahrzeug, Rechner)\\
\hline
\textbf{Team Process:}
\begin{itemize}
\item Reglmäßige Reviews (14 tägig)
\item Optional: Teilnahme am Teammeeting
\end{itemize}\\
\hline
\end{tabular}

\newpage